\thesistitle{Multi-scale Structure-function Analysis of Mitochondrial Network Morphology and Respiratory State in Budding Yeast }

\degreename{Doctor of Philosophy}

% Use the wording given in the official list of degrees awarded by UCI:
% http://www.rgs.uci.edu/grad/academic/degrees_offered.htm
\degreefield{Biomedical Engineering}

% Your name as it appears on official UCI records.
\authorname{Swee Siong Lim}

% Use the full name of each committee member.
\committeechair{Assistant Professor Susanne Rafelski}
\othercommitteemembers
{
 Associate Professor Elliot Botvinick\\
 Professor Suzanne Sandmeyer\\
 Professor Vasan Venugopalan
}

\degreeyear{2015}

\copyrightdeclaration
{
  {\copyright} {\Degreeyear} \Authorname
}

% If you have previously published parts of your manuscript, you must list the
% copyright holders; see Section 3.2 of the UCI Thesis and Dissertation Manual.
% Otherwise, this section may be omitted.
% \prepublishedcopyrightdeclaration
% {
% 	Chapter 4 {\copyright} 2003 Springer-Verlag \\
% 	Portion of Chapter 5 {\copyright} 1999 John Wiley \& Sons, Inc. \\
% 	All other materials {\copyright} {\Degreeyear} \Authorname
% }

% The dedication page is optional.
\dedications
{
To my parents.
}

\acknowledgments
{I would like to express the deepest appreciation to my committee chair, Dr Susanne Rafelski for guiding and supervising me in my journey towards earning this PhD. Without her dedication and  unwavering support the journey could not have been completed, and for that I am eternally indebted.
She provided valuable insight and directed my research so that slowly but surely it took shape and helped me refine it to achieve a standard of scholarship befitting a doctorate project. Her mentoring style was the best of both worlds; she knew when to direct me with more attention when I lacked a clear research focus while still allowing me the freedom to explore and improvise, allowing me to independently tackle my problems.

I would like to thank my committee members, Dr Elliot Botvinick, Dr Suzanne Sandmeyer and Dr Vasan Venugopalan for having the patience and dedication to give me feedback and supervision from the nascent stages of this project all the way to the end. They were very patient even when I was still learning how to formulate and communicate my research goals and were always very supportive in helping me to overcome these difficulties.

I would also like to thank my lab members, Irina Mueller, Vaishali Jayashenkar, Tatsuhisa Tsuboi, Matheus Viana and Maja Bialecka, all of whom have assisted me in some form or other during my doctorate. Whether it was through providing feedback during lab meetings or assistance in their area of expertise, their assistance was vital in helping me complete this journey.

Last but not least, I thank my parents and brother, I could never express my gratitude sufficiently in a few short sentences. They have supported me my entire life and are the spiritual rock that I lean on.

Financial support was provided by the NSF Grant MCB-1330451.
}


% Some custom commands for your list of publications and software.
\newcommand{\mypubentry}[3]{
  \begin{tabular*}{1\textwidth}{@{\extracolsep{\fill}}p{4.5in}r}
    \textbf{#1} & \textbf{#2} \\ 
    \multicolumn{2}{@{\extracolsep{\fill}}p{.95\textwidth}}{#3}\vspace{6pt} \\
  \end{tabular*}
}
\newcommand{\mysoftentry}[3]{
  \begin{tabular*}{1\textwidth}{@{\extracolsep{\fill}}lr}
    \textbf{#1} & \url{#2} \\
    \multicolumn{2}{@{\extracolsep{\fill}}p{.95\textwidth}}
    {\emph{#3}}\vspace{-6pt} \\
  \end{tabular*}
}

% Include, at minimum, a listing of your degrees and educational
% achievements with dates and the school where the degrees were
% earned. This should include the degree currently being
% attained. Other than that it's mostly up to you what to include here
% and how to format it, below is just an example.
\curriculumvitae
{

\textbf{EDUCATION}
  
  \begin{tabular*}{1\textwidth}{@{\extracolsep{\fill}}lr}
    \textbf{Doctor of Philosophy in Biomedical Engineering} & \textbf{2015} \\
    \vspace{6pt}
    University of California, Irvine& \emph{Irvine, California} \\
    \textbf{Master of Science in Biomedical Engineering} & \textbf{2008} \\
    \vspace{6pt}
    University of California, Irvine& \emph{Irvine, California} \\
    \textbf{Master of Engineering in Mechanical Engineering} & \textbf{2001} \\
    \vspace{6pt}
    Imperial College London & \emph{London, United Kingdom} \\
  \end{tabular*}

\vspace{12pt}
\textbf{RESEARCH EXPERIENCE}

  \begin{tabular*}{1\textwidth}{@{\extracolsep{\fill}}lr}
    \textbf{Graduate Research Assistant} & \textbf{2009--2015} \\
    \vspace{6pt}
    University of California, Irvine & \emph{Irvine, California} \\
  \end{tabular*}

\vspace{12pt}
\textbf{TEACHING EXPERIENCE}

  \begin{tabular*}{1\textwidth}{@{\extracolsep{\fill}}lr}
    \textbf{Teaching Assistant} & \textbf{2010--2011} \\
    \vspace{6pt}
    University of California, Irvine & \emph{Irvine, California} \\
  \end{tabular*}

\pagebreak

\textbf{REFEREED JOURNAL PUBLICATIONS}

 \mypubentry{Quantifying mitochondrial content in living cells}{2015}{Methods in Cell Biology,Volume 125, ISSN 0091-679X}

\mypubentry{A quantitative structure-function analysis of mitochondrial network morphology and respiratory state in budding yeast}{in preparation}{}
\vspace{12pt}
\textbf{CONFERENCE PRESENTATIONS}

 \mypubentry{Quantifying the relationship between mitochondrial network topology and bioenergetics in budding yeast}{Dec 2013}{American Society For Cell Biology, New Orleans}\\
 \mypubentry{A quantitative, multi-scale structure-function analysis of mitochondrial network morphology and respiratory state in \emph{Saccharomyces cerevisiae}}{Dec 2015}{American Society For Cell Biology, San Diego}
  

\vspace{12pt}
\textbf{SOFTWARE}

  \mysoftentry{Pipeline software}{https://github.com/moosekaka/sweepython}
  {Repository for all codes used in this dissertation.}

}

% The abstract should not be over 350 words, although that's
% supposedly somewhat of a soft constraint.
\thesisabstract
{Remodeling of the mitochondrial network in response to metabolism involves changes to mitochondrial structure from the ultrastructure to the cellular level. Morphological changes in dysfunctional mitochondria that manifest in diseases such as Parkinson's and Leber's hereditary optic neuropathy drive the need to have a 'systems' level understanding of the relationship between mitochondrial structure and function. However, an integrated, quantitative understanding of the mechanisms linking the changes of structure in response to functional state, and vice versa, is lacking in the field. We developed a multi-scale, quantitative image analysis pipeline and database to simultaneously extract structural features and functional markers of mitochondrial networks for further analysis. We applied this pipeline to the budding yeast, Saccharomyces cerevisiae, which we grew in different carbon sources to achieve distinct respiratory states. Our system was able to quantitatively show that the spatial distribution of mitochondrial membrane potential (ΔΨ, an indicator of mitochondrial function) within individual mitochondrial tubules was nonrandom and dependent on the respiratory state of the cell. These differences were consistent with known alterations to the cristae of the mitochondria. We next investigated the relationship between the connectivity of the mitochondrial network and ΔΨ. Network connectivity is generated by fission and fusion events between individual tubules within the mitochondrial network. Mitochondrial fusion and bioenergetic status are known to be interdependent. We were thus surprised that nowhere in our exhaustive network measurement-based analysis was ΔΨ upregulated in more highly connected networks or network regions as we had predicted. We expect that dynamic data will be needed to detect local regions of the network undergoing fusion dynamics and that it may be these highly dynamic regions of the network that could be be upregulated in ΔΨ. We also investigated the asymmetry of ΔΨ between the mother and daughter bud (future daughter cell) and found that ΔΨ was maintained at a higher level in the daughter throughout the cell division cycle. We detected an increasing gradient in the distribution of ΔΨ along the mother-daughter axis and speculate that this might indicate an increasing ΔΨ-dependent process in the direction of the daughter bud during cell growth.}


%%% Local Variables: ***
%%% mode: latex ***
%%% TeX-master: "thesis.tex" ***
%%% End: ***
