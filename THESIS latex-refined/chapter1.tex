\chapter{Introduction}
\clearpage
\section{Mitochondrial structure}
Since their discovery in the mid 19th century, mitochondria have been one of the most studied organelles in the cell. The name mitochondria belies the morphological description of these organelles: they have a granular appearance, hence the word 'chondros' in Greek and resemble threads, hence the word 'mitos' in Greek \cite{zick_cristae_2009}. It is thus appropriate to begin with an overview of the structural features of this ancient organelle whose study has led to some of the most fundamental understanding of basic biology since its discovery.

The organelle has a tube like cross section with a width on the order of \textasciitilde{\numrange{.5}{1} \si{\micron}}  and a total length ranging from the order of \textasciitilde{\numrange{10}{1000} \si{\micron}}, depending on the cell and organism. Mitochondria are believed to have a bacterial origin \cite{gray_mitochondrial_1999}. The remnants of the bacterial genome found in mitochondria are known as mitochondrial DNA (mtDNA). Only about 1\% of mitochondrial proteins are encoded by the mtDNA. The rest are encoded by the nuclear genome, synthesized in the cytosol and imported into mitochondria via membrane bound translocases. The proteins that are encoded by mtDNA are mainly involved in cellular respiration, in particular those involved in the transfer of electrons and the generation of ATP \cite{mishra_mitochondrial_2014}. 

Mitochondria consist of two membranes; an inner mitochondrial membrane (IMM) which is impermeable to most ions and metabolites enveloping a matrix compartment and an outer membrane (OM) that is freely diffusible to most molecules below \SI{5000}{\dalton}. Within the IMM two distinct domains exist, and inner boundary membrane (IBM) and the cristae membrane. The IBM closely parallels the outer membrane while the cristae forms invaginations of the inner membrane into the matrix space. The cristae membrane possess a distinct substructure known as cristae junctions. These are narrow ring like structures that are believed to subcompartmentalize the inner membrane into the cristae membrane domain and the inner boundary membrane domain \cite{mannella_topology_2001}. Collectively the various membranes and substructures at this scale are known as the ultrastructure of the mitochondria. The inner membrane has an extremely high protein to lipid ratio (75:25) compared to other biological membranes. The reason for this becomes obvious when one realizes that it contains various protein complexes which are involved in oxidative phosphorylation, metabolite exchange, iron-sulfur biogenesis \cite{veatch_mitochondrial_2009}, protein lipid synthesis \cite{osman_making_2011}, import of nuclear encoded proteins required in biogenesis \cite{schmidt_mitochondrial_2010} and remodeling of the network as well as apoptotic \cite{tait_mitochondria_2010} signaling factors. We detail the various functions that involve mitochondria ultrastructure components and beyond in the next section.
\section{Mitochondrial function}
The most well known role of mitochondria as energy production centers of the cell was elucidated by the pioneering work into the respiratory chain, oxidative phosphorylation process (OXPHOS) and chemiosmotic theory in the 50's and 60's \cite{chance_respiratory_1955,mitchell_coupling_1961}. Their role in cellular energy production begins with the oxidation of the small molecule pyruvate, fatty acids and amino acids via the citric acid cycle (TCA) and β-oxidation, both of which take place in the matrix space of mitochondria. The reducing agents (NADH and FADH$_2$) generated from the oxidation of these molecules are reoxidized by donating their electrons to the mitochondrial electron transport chain (ETC) and ultimately reduce oxygen to water. The ETC is concentrated in the cristae membrane domain. The ETC consists of a series of increasingly electronegative enzyme complexes, and the transfer of electrons through these complexes releases energy that is used to translocate protons from the matrix to the intermembrane space (IMS), generating a proton gradient, known as the proton motive force (PMF). The PMF is used by the F1-F0 ATP synthase (complex V) to generate adenosine triphosphate (ATP) in the matrix, which is subsequently transported via an antiporter to the cytosol to power cellular processes. The PMF consists of an electric potential (ΔΨ) and a chemical potential (pH). Inhibition of respiratory chain complexes results in a decrease of ΔΨ \cite{benard_ultrastructure_2008}.

 ΔΨ is an important measure of the functional state of mitochondria. In addition to its role in providing the necessary electrochemical gradient for ATP synthesis, ΔΨ is necessary for the non-bioenergetic related roles that mitochondria are involved in. For example the import of the vast majority of nuclear encoded proteins which are synthesized in the cytosol and translocated into the matrix are known to be dependent on ΔΨ \cite{schmidt_mitochondrial_2010, wiedemann_protein_2004}. ΔΨ also plays a role during remodeling of the mitochondrial network in a process known as mitochondrial dynamics. In addition we have already mentioned that mitochondria are also involved in lipid and iron-sulfur synthesis, cell signaling and apoptosis. These non-bioenergetic related processes, together with those involved in oxidative phosphorylation require changes to the morphology of the organelle at the ultrastructure and larger scales in order to ensure an optimum level of metabolite flux between the various compartments in the cell.
\section{The link between structure and function in mitochondrial remodeling}
Mitochondria provide an illustrative example of how changes to the structure of the organelle are coordinated with the functional state of the cell. When respiration requirements are high, the cristae in mitochondria exhibit a 'condensed' state where the matrix volume is reduced and the cristae volume is enlarged. In contrast, when respiration requirements are low, the matrix space expands and cristae display small volumes ('orthodox state') \cite{hackenbrock_ultrastructural_1968}. It is believed that these changes are to facilitate an optimal diffusion of metabolites during oxidative phosphorylation. In addition to changes that are related to the bioenergetic needs of the cell, mitochondria exhibit structural changes due to apoptotic signaling and biogenesis. During apoptosis, the apoptotic factor cytochrome c is released from the inner cristae compartment into the cytosol \cite{bernardi_cytochrome_1981}. This involves the opening of the cristae junctions via destabilization of OPA1, which is a protein that tethers the cristae junction. The shape of the cristae junction structure is also theorized to form a 'valve' that funnels inward flow of respiratory complexes while inhibiting back diffusion of these complexes back out into the inner boundary membrane, resulting in an enrichment of these complexes in the intra cristae membrane volume \cite{frey_insight_2002}.

The reversible nature of cristae remodeling in response to changes in cellular needs require fission and fusion of the mitochondria \cite{mannella_topology_2001}. Early bioenergetic studies were most often done on isolated mitochondria, hence dynamic changes to the network structure and their importance to the proper functioning of the mitochondrial unit were not appreciated until relatively recently. Since the turn of the century, several important studies have demonstrated in vivo that mitochondrial networks undergo reversible and dynamic changes in response to different energy substrates, providing evidence that mitochondrial structure is intimately linked to its functional state \cite{amchenkova_coupling_1988, jakobs_spatial_2003, meeusen_mitochondrial_2004, rossignol_energy_2004}. Mitochondrial networks in yeast display an increased volume density as a proportion of the total cell volume when grown in non-fermentable carbon sources \cite{stevens_mitochondrial_1981}. Mitochondria in mammalian cells display a hyperfused state when they are exposed to stress factors such as starvation \cite{tondera_slp-2_2009}. In addition to been necessary for reversible cristae remodeling, a crucial role for mitochondrial dynamics is posited by the mitochondrial quality control model. In this model, overall mitochondrial quality is maintained by continual fission and selective fusion of tubules within the network \cite{twig_fission_2008}. In this mechanism segments of mitochondrial tubules are separated (undergo fission) from the network and selectively fuse back to the network according to a threshold level of ΔΨ. The rates of fission and fusion in mitochondrial networks are balanced at steady state \cite{shaw_mitochondrial_2002}. Mitochondrial tubules that are unable to meet this threshold are targeted for disposal via the autophagic machinery, a process known as mitophagy.

During mitosis in mammalian somatic cells, mitochondrial morphology is coordinated with the cell cycle in order to ensure an even distribution of mitochondrial content to daughter cells. During the transition from G2 to M phase, the mitochondrial network changes to a highly fragmented state that is distributed evenly across the soma. This ensures that mitochondria will be evenly distributed to each of the daughter cells after mitosis. In yeast, mitochondria are transported to the bud via the actin cytoskeleton and the mitochondrial content in the bud is actively monitored \cite{rafelski_mitochondrial_2012}.
\section{Pathological consequences of mitochondrial damage}
Mitochondrial quality control primarily serves to protect the overall health of the network by mitigating accumulation of damaged components due to reactive oxygen species (ROS) generated during the OXPHOS process. ROS is generated from excess electrons at the complexes located in the ETC. While some endogenous levels of ROS is normal and plays a role in cell signaling \cite{balaban_mitochondria_2005} excessive ROS production can result in oxidative damage, alterations to the mtDNA, reduced ΔΨ as well as other cell dysfunction. Oxidative degradation of cellular proteins, lipids and DNA by ROS are theorized to be a driver of the aging process \cite{harman_aging:_1956} and associated degenerative diseases \cite{nunnari_mitochondria:_2012}. It has been proposed that aged related diseases result from somatic accumulation of damaged mtDNA which are exposed to much higher levels of ROS due to its proximity to the respiratory complexes \cite{wallace_mitochondrial_2005}. Studies in mice heterozygous for a mitochondrial matrix located antioxidant enzyme, MnSOD (superoxide dismutase) had lifelong reduced levels of these enzymes and showed increased oxidative damage to nuclear DNA and a four fold increase in tumor rates \cite{remmen_life-long_2003}. These mice had reduced liver cell ΔΨ and respiration levels along with greatly increased levels of lipid peroxidation damage \cite{kokoszka_increased_2001}. The morphology of mitochondria exposed to oxidative stress tend to show a fragmented and swollen shape, and their ultrastructure display fewer numbers of cristae \cite{jendrach_short_2008}.

Another clinically relevant example of how damaged mitochondria result in pathogenic signaling is in defects to the mitochondrial autophagy machinery. Mitochondria that accumulate excessive damage are normally targeted for autophagy by accumulation of the protein PINK1 on the outer membrane. PINK1 then recruits the cytosolic protein Parkin, which then induces a series of pathways that ultimately result in elimination of the damaged mitochondria by the autophagy machinery of the cell, a process also known as mitophagy. Mutations to PINK1 and Parkin have been shown to lead to neuronal and muscle cell degeneration, accumulation of defective mitochondria and early onset Parkinson's disease \cite{valente_hereditary_2004,youle_mitochondrial_2012}. Cells treated with a siRNA knockdown for in PINK1 also display an aberrant, fragmented morphology, and their cristae were much less numerous than in untreated cells \cite{exner_loss-function_2007}.
 
Mitochondria in mammals are exclusively inherited from the mother. Because of the lack of genetic recombination and high mutation rates from ROS exposure, mechanisms have evolved to limit the distribution of mtDNA with pathogenic mutations to offsprings. However when these control mechanisms fail, the offspring inherits a significant load of harmful mutations in their mtDNA and experience a host of diseases related to OXPHOS defects. These diseases, termed encephalomyopathies display a broad range of phenotypes that are most obvious in tissues with high metabolic needs such muscle, nerve and brain tissue. An example of a disease in this class is Leber's hereditary optic neuropathy (LHON), which is characterized by a loss of central vision \cite{carelli_retinal_2009}.
\section{Motivation and goal of thesis}
Remodeling of the mitochondrial network in response to metabolic changes entails changes to the structure of the mitochondria from the ultrastructure level (cristae), network level (connectivity between tubules) up to the cellular level. Morphological changes in dysfunctional mitochondria that are manifest in the pathologies mentioned previously (which is by no means complete) drives the need to have a 'systems' level understanding of the relationship between mitochondrial structure and function. However, an integrated, quantitative understanding of the mechanisms linking the changes of structure in response to functional state (and vice versa) is lacking in the field. Previous studies have provided detailed investigations at a specific level (for example detailed studies of the ultrastructure via electron microscopy, \cite{vogel_dynamic_2006,bornhovd_mitochondrial_2006}) without considering the context of changes at the macroscopic level (network and cell). Other studies have described qualitatively the phenotype of changes in mitochondrial morphology in response to alterations of growth conditions and mitochondrial fusion machinery \cite{jakobs_spatial_2003,sesaki_division_1999}. At the cellular level past studies have shown an asymmetry of function between mother and daughter cells in budding yeast, without an analysis of the spatial distribution of this asymmetry \cite{aguilaniu_asymmetric_2003,laun_aged_2001}. A more fundamental requirement that was lacking until very recently is a method to characterize mitochondrial structure beyond visual inspection and qualitative assessment. While advances in this field has made it possible to analyze mitochondrial structure quantitatively \cite{rafelski_mitochondrial_2012, sukhorukov_emergence_2012, vowinckel_mitoloc:_2015}, no study has been presented so far that investigates the relationship between function and network structure in an integrated and statistically validated manner.

\emph{The goal of this thesis is to develop a multi-scale, quantitative investigation of the changes in structural features and functional states of yeast mitochondrial networks in response to changes to the metabolic state of the cell.}
\section{Budding yeast as a model organism for studying structure-function relationship}
The metabolic states that we are interested in are those that minimize and maximize the contribution of OXPHOS derived respiration to the energetic needs of the cell. In this regard, the budding yeast \textit{Saccharomyces cerevisiae} model organism confers particular advantages with regards to the specificity and ease of manipulation of cellular respiration states in order to study changes to the mitochondrial network structure. By simply switching the carbon source of their growth media, one can either induce cells to undergo only fermentation (when grown in high glucose concentrations, which also represses mitochondrial growth), both fermentation and respiration (using a fermentable non repressing substrate) or maximal aerobic respiration (when grown in a non-fermentable substrate such as glycerol). In addition powerful genetic methods have been developed for the isolation of mutant strains lacking mitochondrial dynamics \cite{sesaki_division_1999,kanki_atg32_2009} and inheritance related machinery \cite{itoh_complex_2002, itoh_mmr1p_2004}. Lastly, mitochondria in yeast have networks that are much smaller compared to mammalian cells. However they still retain complex three-dimensional structures. This makes it much easier to segment, quantify and analyze the networks using the full power of network theory, allowing one to study the relationship between network structure and function quantitatively. 
\section{Overview of thesis}
The following is a brief summary of the work done and results obtained in Chapters \ref{ch:two}--\ref{ch:six} of this thesis and how they are relevant towards meeting the goal of this thesis.

In \Fref{ch:two}, we detail a method to map a parameter for mitochondrial function (ΔΨ) to structure using a mitochondrial dye for ΔΨ and a fluorescent protein targeted to the mitochondrial matrix. The three dimensional spatial information containing functional, structural and other parameters are stored in a database that allows one to easily quantify and compare these parameters at different size scales, allowing one to have a truly integrated view of the structural and functional changes that occur in yeast mitochondrial networks in response to changes in respiration state. This database is subsequently used for the analysis done in \Fref{ch:four}, \ref{ch:five} and \ref{ch:six}. We wanted to compare changes to structure of the mitochondrial network in response to metabolic need. In order to do this, in \Fref{ch:three} we measured the oxygen consumption rate of cells growing in different carbon sources. Based on the literature and previous work we had some idea for the expected respiration rate and their corresponding ΔΨ level. Unexpectedly we found that the oxygen consumption rate and ΔΨ did not have a simple correlation for some of the carbon sources, therefore we concluded that mitochondrial respiration efficiency depended on which carbon source it was in. We also discussed why we would need a few additional parameters using a different instrument in order to confirm our reasoning. 

Having developed the computational tools and a set of metabolic conditions to compare structure function changes, in \Fref{ch:four} we proceeded to investigate the distribution of ΔΨ in a single mitochondrial tubule. We compared the distribution of real ΔΨ with randomly sampled ΔΨ in mitochondrial tubules to establish a baseline level of heterogeneity and concluded that the observed heterogeneity was real and nonrandom. Furthermore we found that tubules in respiratory conditions correlated at smaller length scales and reasoned that it is most likely due to an increase in cristae density. We also found that tubules in respiratory conditions had an increased tubule thickness while having a lower variation of thickness along the length of the tubule. In \Fref{ch:five}, we analyzed the distribution of function (ΔΨ) within regions of the mitochondrial network. We found that connectivity of the mitochondrial network scaled with surface density, but function (ΔΨ) did not scale with connectivity. From these results we concluded that our assumption that the network connectivity could indicate regions of increased fusion activity (along with increased ΔΨ) was wrong, or at the very least our static data could not distinguish sites of higher fusion activity. Furthermore mitochondrial fragments that were isolated from the mitochondrial network proper did not show any difference in their ΔΨ levels. Again we concluded that we would need dynamic data or a mutant strain for mitophagy in order to obtain a conclusive result. In \Fref{ch:six}, we detail the development of a method to analyze the spatial distribution of ΔΨ between the mother and buds of budding yeast based on the functional mapping pipeline developed in \Fref{ch:two}. Similar to previous studies, we found evidence for a functional asymmetry between mother and bud (buds have a higher ΔΨ level).  In addition we showed that this functional asymmetry was maintained throughout the entire cell division cycle. We also detected an increasing gradient of ΔΨ in the direction of the bud along the mother-bud cellular axis. Lastly in \Fref{ch:seven} we conclude with the contributions and significance of this thesis and provide some future directions for extending the research work done.




%%% Local Variables: ***
%%% mode: latex ***
%%% TeX-master: "thesis.tex" ***
%%% End: ***
